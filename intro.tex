\section{Introduction}
%%% Machine learning %%%
%
Machine learning is a powerful methodology to construct efficient and robust
predictors and classifiers.
%
In particular, deep learning, a technology to train complicated neural
networks with multiple hidden layers, i.e., deep neural networks (DNNs),
has been attracted much attention because of its outstanding performance in
several artificial intelligence tasks \cite{krizhevsky2012imagenet,seide2011conversational,DeepFace}.
%
Now, literature suggests the ability of supervised learning not only to reproduce
``intuition and experience'' based on human supervision \cite{DeepFace},
but also to successfully classify the objects that humans cannot
sufficiently classify with inspection alone \cite{Horikawa2013,Mural1991}.
%
This work is motivated by the cases in which the supervised
classifier eclipses the human decisions.
% END OF PARAGRAPH

%%% Superiority of machines %%%
%
In such cases, one might be able to learn the knowledge of the data from
the trained classifiers.
%
We may say that the case that the classifiers defeat the human
occurs when the classifier holds more knowledge about the
data and the subject classes than us, because our incompetence in the classification problems
can be attributed solely to our lack of understanding about the class
properties and/or the similarity metrics.
%
For example, Miyawaki et al.\ \cite{miyawaki2008visual} used
logistic regression for this purpose in neuroimaging.
%
Let us consider the classifier $f:\RR^{d} \rightarrow [0, 1]$ with the
$d$-dimensional input $\bvec{x}$.
Logistic regression is a linear model
\begin{equation}
 f(\bvec{x}) = \sigma \left(\sum_{i}^{d} w_i x_i + b\right),
\end{equation}
where $w_i$ and $b$ denote a
weight parameter and a bias parameter, respectively, and $\sigma$ is a sigmoid function.
%
Miyawaki et al.\ assigned each small three-dimensional brain unit
(voxel) to a input dimension,
and visualized the determined weights, $(w_1, \dots, w_d)$
as brain map in order to extract the knowledge captured by the trained classifier.
%
One can understand that if the absolute value of weight $|w_i|$ is large enough, then the
corresponding $i$-th input dimension (voxel) is important for characterizing the
subject class.
%
Other neuroimaging studies \cite{LaConte2005,yamashita2008sparse,abraham2014machine} also
provided such visualization of linear models.
% END OF PARAGRAPH

%%% Machines are complicated %%%
%
Unfortunately, however, trained classifiers are often so complex that
they defy the human interpretation.
%
If one add a hidden layer to the logistic regression:
\begin{equation}
 \label{eq:simple_nn}
f(\bvec{x}) = \sigma \left(\sum_{i}^{l} w_{i}^{(2)} h
\left(\sum_{j}^{d} w_{ij}^{(1)} x_{j} + b_{i}^{(1)}\right) +
b^{(2)}\right),
\end{equation}
where $h$ is a nonlinear activation function,
then the nonlinear classifier $f$ becomes too complex for the human to
directly analyze the relationship between the input data and the classes,
because the hierarchical architecture of the classifier $f$ masks the
direct connection between the input and output.
%
The equation \eqref{eq:simple_nn} is the simplest version of
feed-forward neural networks, and thus DNNs with more hidden layers are
too complicated for the human to interpret their decision.
%
However, the superiority of nonlinear machine learning techniques like deep
learning \cite{krizhevsky2012imagenet,seide2011conversational,DeepFace} strongly suggests that the trained classifiers
capture the ``invisible'' properties of the subject classes.
%
Geoff Hinton solidified this into the philosophy of
``dark knowledge'' captured within the trained neural networks \cite{DarkKnowledge}.
%
Thus, our objective is to visualize the decision of such nonlinear and hierarchical machines like DNN
in the interpretable form.
% END OF PARAGRAPH

%%% Previous work %%%
For the visualization of high-dimensional feature space of nonlinear machines,
Zurada et al.\ \cite{Zurada1994,Zurada1997},  and Kjems et al.\ \cite{Kjems2002}
presented seminal works.
%
Zurada et al.\ defined \textit{sensitivity} by
\begin{equation}
 \label{eq:sensitivity_intro}
 s_i := E_q \left[ \left(\frac{\partial f(\bvec{x})}{\partial x_i} \right)^2 \right],
\end{equation}
where $q$ is the true distribution of $\bvec{x}$,
in order to ``delete unimportant data components for feed-forward neural networks.''
Kjems et al.\ visualized Zurada's idea as \textit{sensitivity map}, $\left(s_1, \dots, s_d\right)$
in the context of neuroimaging.
%
Note that when the classifier $f$ is a linear model like logistic
regression, the sensitivity satisfies $s_i \propto w_i^2$.
%
Rasmussen et al.\ \cite{Rasmussen2011} applied this \textit{sensitivity analysis}
to a nonlinear kernel method in functional neuroimaging.
%
Because the sensitivity analysis provides a single map by the
integration \eqref{eq:sensitivity_intro} over the
whole input space $\RR^{d}$, we can regard the sensitivity map as a
``global'' visualization of the classifier $f$.
%
Baehrens et al.\ \cite{Baehrens2010} took another approach.
%
Baehrens et al.\ defined \textit{local explanation vector} by the
gradient $\nabla f (\bvec{x})$ and used it for
analyzing the importance of features at each ``localized'' point in the
input space.
%
Although these efforts have been made to visualize the
classifiers,
there is still much room left for improvement.
%
Machine learning techniques in neuroimaging, for example,
prefer linear kernels to nonlinear kernels because of the lack of
visualization techniques \cite{LaConte2005}.
% END OF PARAGRAPH

%%% Methods %%%
In this study, we attempt to  generalize the idea of sensitivity
analysis, and develop a new framework that aids us in extracting the
knowledge from classifiers that are trained in a supervised manner.
%
Our framework is superior to the predecessors in that it can:
\begin{enumerate}
 \item be used to identify a pair of discriminative input
       features that act oppositely in characterizing a class,
 \item identify \textit{combinations} of discriminative features that strongly
       characterize the subject classes,
 \item provide platform for developing sparse, visually intuitive
       sensitivity maps.
\end{enumerate}
%
The new framework gives rise to the algorithm that we refer
to as ``Principal Sensitivity Analysis (PSA),'' which is analogous to
the well-established ``Principal Component Analysis (PCA).''
%
To demonstrate the PSA's ability to extract the knowledge captured by
the trained classifiers, we applied the PSA to the classifier trained
for digit classification.
% END OF PARAGRAPH

%%% Application to fMRI %%%
As a practical application of PSA, we next used the PSA for functional neuroimaging.
%
In order to capture the discriminative features of brain activities
common to all human subjects,
we trained a DNN with a large-scale neuroimaging database and confirmed
that, in comparison to other baseline methods, the DNN captured more discriminative features.
%
We then applied the PSA to the DNN-based decoder to highlight the subject-independent
features used by the decoders for classifying brain activities.
%
By illustrating these features as brain maps, we found the association between
the features and functional connectivity established in human fMRI studies \cite{raichle2001default,raichle2007default,taylor2009two,cole2013multi}.
% END OF PARAGRAPH

%%% OUTLINE %%%
This thesis is structured as follows.
%
In Section\,\ref{sec:methods}, we will provide the PSA's algorithm along with
the explanation of the conventional sensitivity analysis.
%
In Section\,\ref{sec:NN}, we will explain the architecture of
 feed-forward neural networks, to which we apply the PSA, and
 also explain its training.
%
In Section\,\ref{sec:results}, we will show a primitive application of
the PSA to the classifier trained for digit classification,
and demonstrate its advantages in comparison to the conventional sensitivity analysis.
%
In Section\,\ref{sec:application}, we will also provide an practical
application of the PSA to functional neuroimaging.
%
Finally, in Section\,\ref{sec:discussion}, we will summarize the results
and make conclusive remarks.
% END OF PARAGRAPH
