\begin{abstract}
 In this thesis, we studied methods to visualize the knowledge captured by
 supervised classifiers.
 %
 In particular, we developed a new method, ``Principal Sensitivity
 Analysis (PSA),'' to analyze the sensitivity of the trained classifier.
 %
 In PSA, principal sensitivity map (PSM) is defined as the
 direction in the input space to which the classifier is most sensitive,
 and $k$-th PSM is also analogously defined for each $k$.
 %
 Using these maps, PSA decomposes the input space based on the sensitivity of
 the classifier.
 %
 As a primitive case study, we first applied the PSA to the classifier trained
 for digit classification.
 %
 We were able to find a direct association between the PSMs and the
 discriminative features of the digits that we humans intuitively use
 for classification.
 %
 Next, in order to assess the performance of
 our algorithm on nonlinear and hierarchical classifiers in a
 practical setting, we applied the PSA to the deep neural network (DNN) trained with
 large-scale neuroimaging database.
 %
 We confirmed that, in comparison to other baseline methods,
 the DNN can capture richer discriminative features of brain activities
 that are common to many human subjects.
 %
 Interestingly, we were able to find nontrivial connections between
 the PSMs of the trained DNN and the functional connectivity among brain areas,
 a phenomenon studied in neuroscience.
 %
 This suggests a relation between the discriminative features of
 brain activities and the functional connectivity.
\end{abstract}